%%%%%%%%%%%%%%%%%%%%%%%%%%%%%%%%%%%%%%%%%
% a0poster Landscape Poster
% LaTeX Template
% Version 1.0 (22/06/13)
%
% The a0poster class was created by:
% Gerlinde Kettl and Matthias Weiser (tex@kettl.de)
% 
% This template has been downloaded from:
% http://www.LaTeXTemplates.com
%
% License:
% CC BY-NC-SA 3.0 (http://creativecommons.org/licenses/by-nc-sa/3.0/)
%
%%%%%%%%%%%%%%%%%%%%%%%%%%%%%%%%%%%%%%%%%

%----------------------------------------------------------------------------------------
%	PACKAGES AND OTHER DOCUMENT CONFIGURATIONS
%----------------------------------------------------------------------------------------

\documentclass[a0,landscape]{a0poster}

\usepackage{multicol} % This is so we can have multiple columns of text side-by-side
\columnsep=100pt % This is the amount of white space between the columns in the poster
\columnseprule=3pt % This is the thickness of the black line between the columns in the poster

\usepackage[svgnames]{xcolor} % Specify colors by their 'svgnames', for a full list of all colors available see here: http://www.latextemplates.com/svgnames-colors

\usepackage{times} % Use the times font
%\usepackage{palatino} % Uncomment to use the Palatino font

\usepackage{graphicx} % Required for including images
\graphicspath{{figures/}} % Location of the graphics files
\usepackage{booktabs} % Top and bottom rules for table
\usepackage[font=small,labelfont=bf]{caption} % Required for specifying captions to tables and figures
\usepackage{amsfonts, amsmath, amsthm, amssymb} % For math fonts, symbols and environments
\usepackage{wrapfig} % Allows wrapping text around tables and figures

%------------------------
% xelatex
\usepackage{fontspec}
\usepackage{xunicode}
\usepackage{xltxtra}

% languages
\usepackage{fixlatvian}
\usepackage{polyglossia}
\setdefaultlanguage{latvian}
%\setotherlanguages{english,russian}

% bibliography
%\usepackage{csquotes}
\usepackage[
    backend=biber,
    style=numeric-comp,
    sorting=none,
    natbib=true,
    url=false,
    doi=true%,
    %eprint=false
]{biblatex}
\addbibresource{bibliography.bib}


% New commands
\newcommand{\keywordname}{Atslēgas vārdi:}
\newcommand{\keywords}[1]{\par\addvspace\baselineskip\noindent\keywordname\enspace\ignorespaces#1}

\begin{document}

%----------------------------------------------------------------------------------------
%	POSTER HEADER 
%----------------------------------------------------------------------------------------

% The header is divided into three boxes:
% The first is 55% wide and houses the title, subtitle, names and university/organization
% The second is 25% wide and houses contact information
% The third is 19% wide and houses a logo for your university/organization or a photo of you
% The widths of these boxes can be easily edited to accommodate your content as you see fit

\begin{minipage}[b]{0.55\linewidth}
\veryHuge \color{NavyBlue} \textbf{Neironu tīkli un nepārtrauktas darbību telpas Markova izvēles procesi} \color{Black}\\ % Title
\Huge\textit{Maģistra kursa darbs}\\[1cm] % Subtitle
\huge \textbf{Rihards Krišlauks}\\ % Author(s)
\huge Latvijas Universitātes Datorikas fakultāte\\ % University/organization
\end{minipage}
%
\begin{minipage}[b]{0.25\linewidth}
\color{DarkSlateGray}\Large \textbf{Contact Information:}\\
Department Name\\ % Address
University Name\\
123 Broadway, State, Country\\\\
Phone: +1 (000) 111 1111\\ % Phone number
Email: \texttt{john@LaTeXTemplates.com}\\ % Email address
\end{minipage}
%
\begin{minipage}[b]{0.19\linewidth}
\includegraphics[width=20cm]{lu-logo-full.png} % Logo or a photo of you, adjust its dimensions here
\end{minipage}

\vspace{1cm} % A bit of extra whitespace between the header and poster content

%----------------------------------------------------------------------------------------

\begin{multicols}{4} % This is how many columns your poster will be broken into, a poster with many figures may benefit from less columns whereas a text-heavy poster benefits from more

%----------------------------------------------------------------------------------------
%	ABSTRACT
%----------------------------------------------------------------------------------------

\color{Navy} % Navy color for the abstract

\begin{abstract}

Neironu tīklu lietojums praksē un literatūrā ir parādījis to pozitīvās īpašības, kā robustumu, spēju vispārināt un pielietojuma iespēju daudzveidību.
Darbā tiek pētīts neironu tīklu lietojums stimulētās mācīšanās paradigmā, ar mērķi pētīt iespējas to labās īpašības pārnest uz šo nozari, īpašu uzmanību pievēršot tieši to pielietojumam nepārtrauktu darbības telpu Markova izvēles procesos.
Autors iztirzā literatūrā parādīto klasisko pieeju un algoritmu atsevišķu komponenšu sniegumu nozīmīgākajos aspektos, kas saistīti ar to lietojumu nepārtrauktās darbību telpās.
Izpētes gaitā tiek nonākts līdz continuous action-critic learning automaton (CACLA) algoritmam, kas pārvar problēmas, ar ko saskaras citi apskatītie algoritmi un pieejas gan lietojamībā nepārtrauktās darbību telpās, gan savietojamībā ar neironu tīkliem, kā arī tiek secināts, ka tā darbībā ir vairākas citas pozitīvas īpašības.
%Darbs tiek noslēgts ar diskusiju par 

\keywords{stimulētā mācīšanās; neironu tīkli; Markova izvēles procesi; nepārtrauktas telpas.}

\end{abstract}

%----------------------------------------------------------------------------------------
%	INTRODUCTION
%----------------------------------------------------------------------------------------

\color{SaddleBrown} % SaddleBrown color for the introduction

\section*{Ievads}

Salīdzinot mašīnmācīšanās disciplīnas, stimulētā mācīšanās uz pārējo fona izceļas.
Tā stāv ļoti tuvu intuitīvajai izpratnei, par to, kas vispār ir mācīšanās tādā nozīmē, kā to dara dzīvnieki vai cilvēks.
Tās pamatā ir aģenta mijiedarbība ar apkārtējo vidi, lai patstāvīgi iemācītos tajā darboties, atbilstoši kādai izpratnei par optimālumu.
Kā jau matemātiskā disciplīnā tajā apskatāmās situācijas tiek formalizētas, un parasti tas tiek darīts ar Markova izvēles procesu palīdzību.
Tas ļauj stimulētās mācīšanās paradigmā formulēt ļoti plašu problēmu loku, sākot ar, piemēram, orientēšanos labirintā, līdz autonomai lidaparāta kontrolei.

Uzdevumu risināšana nepārtrauktās darbību telpās ir plaši pētīta, un to var veikt samērā sekmīgi.
Tomēr citāda situācija paveras nepārtrauktu darbības telpu gadījumā.
Šāda tipa uzdevumu risināšana ir mazāk pētīta, un plaši lietotās nepārtrauktu stāvokļu telpu pieejas šeit nav tiešā veidā izmantojamas.

No otras puses, literatūrā un praksē ir sastopams daudz piemēru, kas izceļ neironu tīklus kā universālu līdzekli dažāda tipa problēmu risināšanā.
Tiek demonstrētas to pozitīvās īpašības, kā robustums, spēja vispārināt un pielietojuma iespēju daudzveidība.
Ir zināms, ka neironu tīkli var kalpot kā universāls nepārtrauktu funkciju aproksimācijas līdzeklis.
Tas liek domāt, ka neironu tīkli ir dabīgā veidā izmantojami stimulētās mācīšanās uzdevumos aģenta darbību izvēles mehānisma reprezentācijai.
To var intuitīvi pamatot ar neironu tīklu spēju atrast sakarības datos -- virspusēji nav redzami ierobežojumi, lai šī īpašība nebūtu pārnesama arī stimulētās mācīšanās uzdevumos, sakarību meklēšanai aģenta kontroles stratēģijā.
Turklāt, ir pamats domāt, ka rezultātā iegūtā algoritma uzbūve būtu gana vispārīga, lai to kā universālu līdzekli pielietotu plašam problēmu lokam.

%----------------------------------------------------------------------------------------
%	OBJECTIVES
%----------------------------------------------------------------------------------------

\color{DarkSlateGray} % DarkSlateGray color for the rest of the content

\section*{Galvenie mērķi}

Par darba mērķi tiek izvirzīts izvērtēt iespējas risināt stimulētās mācīšanās uzdevumus nepārtrauktās darbības telpās, izmantojot neironu tīklus kā līdzekli stratēģijas reprezentēšanai.
Meklētas tiek vispārīgas pieejas, problēmas risināšanai, kas ļautu nonākt pie universāla un viegli pielietojama algoritma stimulētās mācīšanās uzdevumu risināšanai nepārtrauktās darbību telpās.

%----------------------------------------------------------------------------------------
%	MATERIALS AND METHODS
%----------------------------------------------------------------------------------------

\section*{Markova izvēles procesi}

Markova izvēles procesi (angliski Markov decision processes, turpmāk tekstā - MDP) formalizē un ļauj modelēt izvēles veikšanas procesu apstākļos, kur darbības rezultāts ir atkarīgs tikai no sistēmas pašreizējā stāvokļa, bet ir daļēji nejaušs, t.i., izvēles veicējs procesu kontrolē tikai daļēji.
Mērķis ir kontrolēt sistēmu tā, lai tiktu maksimizēta kāda metrika, kas ir atkarīga no katrā solī veiktās darbības rezultāta.
Tiek uzskatīts, ka MDP ir ieviesti \autocite{Bel}.

%------------------------------------------------

\subsection*{Mathematical Section}

Nulla vel nisl sed mauris auctor mollis non sed. 

\begin{equation}
E = mc^{2}
\label{eqn:Einstein}
\end{equation}

Curabitur mi sem, pulvinar quis aliquam rutrum. (1) edf (2)
, $\Omega=[-1,1]^3$, maecenas leo est, ornare at. $z=-1$ edf $z=1$ sed interdum felis dapibus sem. $x$ set $y$ ytruem. 
Turpis $j$ amet accumsan enim $y$-lacina; 
ref $k$-viverra nec porttitor $x$-lacina. 

Vestibulum ac diam a odio tempus congue. Vivamus id enim nisi:

\begin{eqnarray}
\cos\bar{\phi}_k Q_{j,k+1,t} + Q_{j,k+1,x}+\frac{\sin^2\bar{\phi}_k}{T\cos\bar{\phi}_k} Q_{j,k+1} &=&\nonumber\\ 
-\cos\phi_k Q_{j,k,t} + Q_{j,k,x}-\frac{\sin^2\phi_k}{T\cos\phi_k} Q_{j,k}\label{edgek}
\end{eqnarray}
and
\begin{eqnarray}
\cos\bar{\phi}_j Q_{j+1,k,t} + Q_{j+1,k,y}+\frac{\sin^2\bar{\phi}_j}{T\cos\bar{\phi}_j} Q_{j+1,k}&=&\nonumber \\
-\cos\phi_j Q_{j,k,t} + Q_{j,k,y}-\frac{\sin^2\phi_j}{T\cos\phi_j} Q_{j,k}.\label{edgej}
\end{eqnarray} 

Nulla sed arcu arcu. Duis et ante gravida orci venenatis tincidunt. Fusce vitae lacinia metus. Pellentesque habitant morbi. $\mathbf{A}\underline{\xi}=\underline{\beta}$ Vim $\underline{\xi}$ enum nidi $3(P+2)^{2}$ lacina. Id feugain $\mathbf{A}$ nun quis; magno. Fusce convallis rutrum turpis, quis aliquet enim accumsan id. Vestibulum ullamcorper porttitor convallis. Integer sagittis interdum malesuada. Class aptent taciti sociosqu ad litora torquent per conubia nostra, per inceptos himenaeos. Sed adipiscing tristique orci at ullamcorper. Morbi accumsan, urna et porttitor pulvinar, lacus risus dignissim massa. Proin sollicitudin. Pellentesque eget orci eros. Fusce ultricies, tellus et pellentesque fringilla, ante massa luctus libero, quis tristique purus urna nec nibh.

%----------------------------------------------------------------------------------------
%	RESULTS 
%----------------------------------------------------------------------------------------

\section*{Results}

Donec faucibus purus at tortor egestas eu fermentum dolor facilisis. Maecenas tempor dui eu neque fringilla rutrum. Mauris \emph{lobortis} nisl accumsan. Aenean vitae risus ante. Pellentesque condimentum dui. Etiam sagittis purus non tellus tempor volutpat. Donec et dui non massa tristique adipiscing.
%
\begin{wraptable}{l}{12cm} % Left or right alignment is specified in the first bracket, the width of the table is in the second
\begin{tabular}{l l l}
\toprule
\textbf{Treatments} & \textbf{Response 1} & \textbf{Response 2}\\
\midrule
Treatment 1 & 0.0003262 & 0.562 \\
Treatment 2 & 0.0015681 & 0.910 \\
Treatment 3 & 0.0009271 & 0.296 \\
\bottomrule
\end{tabular}
\captionof{table}{\color{Green} Table caption}
\end{wraptable}
%
Phasellus imperdiet, tortor vitae congue bibendum, felis enim sagittis lorem, et volutpat ante orci sagittis mi. Morbi rutrum laoreet semper. Morbi accumsan enim nec tortor consectetur non commodo nisi sollicitudin. Proin sollicitudin. Pellentesque eget orci eros. Fusce ultricies, tellus et pellentesque fringilla, ante massa luctus libero, quis tristique purus urna nec nibh.

Nulla ut porttitor enim. Suspendisse venenatis dui eget eros gravida tempor. Mauris feugiat elit et augue placerat ultrices. Morbi accumsan enim nec tortor consectetur non commodo. Pellentesque condimentum dui. Etiam sagittis purus non tellus tempor volutpat. Donec et dui non massa tristique adipiscing. Quisque vestibulum eros eu. Phasellus imperdiet, tortor vitae congue bibendum, felis enim sagittis lorem, et volutpat ante orci sagittis mi. Morbi rutrum laoreet semper. Morbi accumsan enim nec tortor consectetur non commodo nisi sollicitudin.

\begin{center}\vspace{1cm}
\includegraphics[width=0.8\linewidth]{placeholder}
\captionof{figure}{\color{Green} Figure caption}
\end{center}\vspace{1cm}

In hac habitasse platea dictumst. Etiam placerat, risus ac.

Adipiscing lectus in magna blandit:

\begin{center}\vspace{1cm}
\begin{tabular}{l l l l}
\toprule
\textbf{Treatments} & \textbf{Response 1} & \textbf{Response 2} \\
\midrule
Treatment 1 & 0.0003262 & 0.562 \\
Treatment 2 & 0.0015681 & 0.910 \\
Treatment 3 & 0.0009271 & 0.296 \\
\bottomrule
\end{tabular}
\captionof{table}{\color{Green} Table caption}
\end{center}\vspace{1cm}

Vivamus sed nibh ac metus tristique tristique a vitae ante. Sed lobortis mi ut arcu fringilla et adipiscing ligula rutrum. Aenean turpis velit, placerat eget tincidunt nec, ornare in nisl. In placerat.

\begin{center}\vspace{1cm}
\includegraphics[width=0.8\linewidth]{placeholder}
\captionof{figure}{\color{Green} Figure caption}
\end{center}\vspace{1cm}

%----------------------------------------------------------------------------------------
%	CONCLUSIONS
%----------------------------------------------------------------------------------------

\color{SaddleBrown} % SaddleBrown color for the conclusions to make them stand out

\section*{Conclusions}

\begin{itemize}
\item Pellentesque eget orci eros. Fusce ultricies, tellus et pellentesque fringilla, ante massa luctus libero, quis tristique purus urna nec nibh. Phasellus fermentum rutrum elementum. Nam quis justo lectus.
\item Vestibulum sem ante, hendrerit a gravida ac, blandit quis magna.
\item Donec sem metus, facilisis at condimentum eget, vehicula ut massa. Morbi consequat, diam sed convallis tincidunt, arcu nunc.
\item Nunc at convallis urna. isus ante. Pellentesque condimentum dui. Etiam sagittis purus non tellus tempor volutpat. Donec et dui non massa tristique adipiscing.
\end{itemize}

\color{DarkSlateGray} % Set the color back to DarkSlateGray for the rest of the content

%----------------------------------------------------------------------------------------
%	FORTHCOMING RESEARCH
%----------------------------------------------------------------------------------------

\section*{Forthcoming Research}

Vivamus molestie, risus tempor vehicula mattis, libero arcu volutpat purus, sed blandit sem nibh eget turpis. Maecenas rutrum dui blandit lorem vulputate gravida. Praesent venenatis mi vel lorem tempor at varius diam sagittis. Nam eu leo id turpis interdum luctus a sed augue. Nam tellus.

 %----------------------------------------------------------------------------------------
%	REFERENCES
%----------------------------------------------------------------------------------------

%\nocite{*} % Print all references regardless of whether they were cited in the poster or not
%\bibliographystyle{plain} % Plain referencing style
%\bibliography{sample} % Use the example bibliography file sample.bib

\printbibliography

%----------------------------------------------------------------------------------------
%	ACKNOWLEDGEMENTS
%----------------------------------------------------------------------------------------

\section*{Acknowledgements}

Etiam fermentum, arcu ut gravida fringilla, dolor arcu laoreet justo, ut imperdiet urna arcu a arcu. Donec nec ante a dui tempus consectetur. Cras nisi turpis, dapibus sit amet mattis sed, laoreet.

%----------------------------------------------------------------------------------------

\end{multicols}
\end{document}